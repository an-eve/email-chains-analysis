%!TEX root = ../dissertation.tex

 In the modern working environment, people are overloaded with data. We collect, store, and process vast amounts of information for business, safety, and legal purposes. Managing this information manually is increasingly challenging, and employees spend a significant amount of time performing repetitive tasks. Consequently, automation offers a promising breakthrough that could substantially reduce costs.
 
 This project leverages data mining on email data, initially focusing on emails with the potential to extend the findings to other areas. We aim to identify patterns within email interactions to automate repetitive tasks such as reading, responding, attaching files, forwarding information, and managing spam. This automation could significantly enhance productivity and user satisfaction by reducing the manual effort involved in email management.

 Our approach involves analyzing a large dataset of business emails to detect recurring interaction patterns. To reconstruct email chains and convert them into structured data, we use a systematic methodology relying on the emails' metadata and content. After automatically processing the texts, we generate embeddings to convert the text into numerical representations. A time-aware distance metric assesses sequence similarity to cluster the emails, revealing potential automation opportunities.
 
 The results demonstrate the feasibility of extracting processes and similar interactions from emails using the proposed solution. This serves as a model pipeline for future projects, where specific steps can be adapted to meet specified task requirements, improve performance, and adapt to other data formats.

\vspace{1cm}
 
\textbf{Key words:} Email chains detection, data mining, text mining, process extraction, process automation, text embedding, sentence embedding, dinamic time wrapping, density-based clustering.